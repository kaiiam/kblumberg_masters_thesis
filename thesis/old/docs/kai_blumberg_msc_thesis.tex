\PassOptionsToPackage{unicode=true}{hyperref} % options for packages loaded elsewhere
\PassOptionsToPackage{hyphens}{url}
\PassOptionsToPackage{dvipsnames,svgnames*,x11names*}{xcolor}
%
\documentclass[]{article}
\usepackage{lmodern}
\usepackage{amssymb,amsmath}
\usepackage{ifxetex,ifluatex}
\usepackage{fixltx2e} % provides \textsubscript
\ifnum 0\ifxetex 1\fi\ifluatex 1\fi=0 % if pdftex
  \usepackage[T1]{fontenc}
  \usepackage[utf8]{inputenc}
  \usepackage{textcomp} % provides euro and other symbols
\else % if luatex or xelatex
  \usepackage{unicode-math}
  \defaultfontfeatures{Ligatures=TeX,Scale=MatchLowercase}
\fi
% use upquote if available, for straight quotes in verbatim environments
\IfFileExists{upquote.sty}{\usepackage{upquote}}{}
% use microtype if available
\IfFileExists{microtype.sty}{%
\usepackage[]{microtype}
\UseMicrotypeSet[protrusion]{basicmath} % disable protrusion for tt fonts
}{}
\IfFileExists{parskip.sty}{%
\usepackage{parskip}
}{% else
\setlength{\parindent}{0pt}
\setlength{\parskip}{6pt plus 2pt minus 1pt}
}
\usepackage{xcolor}
\usepackage{hyperref}
\hypersetup{
            pdftitle={Interconnecting Arctic observatory data through machine-actionable knowledge representation: are ontologies fit for purpose?},
            pdfauthor={Kai Blumberg},
            colorlinks=true,
            linkcolor=Maroon,
            citecolor=Blue,
            urlcolor=cyan,
            breaklinks=true}
\urlstyle{same}  % don't use monospace font for urls
\usepackage[margin=1in]{geometry}
\usepackage{listings}
\newcommand{\passthrough}[1]{#1}
\usepackage{longtable,booktabs}
% Fix footnotes in tables (requires footnote package)
\IfFileExists{footnote.sty}{\usepackage{footnote}\makesavenoteenv{longtable}}{}
\setlength{\emergencystretch}{3em}  % prevent overfull lines
\providecommand{\tightlist}{%
  \setlength{\itemsep}{0pt}\setlength{\parskip}{0pt}}
\setcounter{secnumdepth}{0}
% Redefines (sub)paragraphs to behave more like sections
\ifx\paragraph\undefined\else
\let\oldparagraph\paragraph
\renewcommand{\paragraph}[1]{\oldparagraph{#1}\mbox{}}
\fi
\ifx\subparagraph\undefined\else
\let\oldsubparagraph\subparagraph
\renewcommand{\subparagraph}[1]{\oldsubparagraph{#1}\mbox{}}
\fi

% set default figure placement to htbp
\makeatletter
\def\fps@figure{htbp}
\makeatother

% Contents of listings-setup.tex
\usepackage{xcolor}

\lstset{
    basicstyle=\ttfamily,
    numbers=left,
    keywordstyle=\color[rgb]{0.13,0.29,0.53}\bfseries,
    stringstyle=\color[rgb]{0.31,0.60,0.02},
    commentstyle=\color[rgb]{0.56,0.35,0.01}\itshape,
    numberstyle=\footnotesize,
    stepnumber=1,
    numbersep=5pt,
    backgroundcolor=\color[RGB]{248,248,248},
    showspaces=false,
    showstringspaces=false,
    showtabs=false,
    tabsize=2,
    captionpos=b,
    breaklines=true,
    breakatwhitespace=true,
    breakautoindent=true,
    escapeinside={\%*}{*)},
    linewidth=\textwidth,
    basewidth=0.5em,
}

\title{Interconnecting Arctic observatory data through machine-actionable
knowledge representation: are ontologies fit for purpose?}
\author{Kai Blumberg}
\date{Submission Date 15.03.18}

\begin{document}
\maketitle

{
\hypersetup{linkcolor=}
\setcounter{tocdepth}{3}
\tableofcontents
}
\hypertarget{introduction}{%
\section{Introduction}\label{introduction}}

\hypertarget{rapid-effects-of-climate-change-on-polar-systems}{%
\subsubsection{Rapid effects of climate change on Polar
systems}\label{rapid-effects-of-climate-change-on-polar-systems}}

Anthropogenic green house gas emissions are leading to increased climate
change and weather extremes.

\begin{longtable}[]{@{}ll@{}}
\caption{Compounds serving as algal metabolites.}\tabularnewline
\toprule
purl & label\tabularnewline
\midrule
\endfirsthead
\toprule
purl & label\tabularnewline
\midrule
\endhead
http://purl.obolibrary.org/obo/CHEBI\_17992 & Sucrose\tabularnewline
http://purl.obolibrary.org/obo/CHEBI\_80716 &
aplysiatoxin\tabularnewline
http://purl.obolibrary.org/obo/CHEBI\_90820 &
11(R)-HEPE(1-)\tabularnewline
http://purl.obolibrary.org/obo/CHEBI\_86386 &
3-mercaptopropionate\tabularnewline
http://purl.obolibrary.org/obo/CHEBI\_17754 & Glycerin\tabularnewline
http://purl.obolibrary.org/obo/CHEBI\_17754 & glycerol\tabularnewline
http://purl.obolibrary.org/obo/CHEBI\_16810 &
2-oxoglutarate(2-)\tabularnewline
http://purl.obolibrary.org/obo/CHEBI\_16914 & salicylic
acid\tabularnewline
http://purl.obolibrary.org/obo/CHEBI\_16914 & Salicylic
Acid\tabularnewline
http://purl.obolibrary.org/obo/CHEBI\_16811 & Methionine\tabularnewline
\bottomrule
\end{longtable}

\hypertarget{microbes-and-biogeochemical-cycles}{%
\paragraph{Microbes and Biogeochemical
cycles}\label{microbes-and-biogeochemical-cycles}}

\begin{quote}
The prokaryotic and eukaryotic microorganisms that drive the pelagic
ocean's biogeochemical cycles are currently facing an unprecedented set
of comprehensive anthropogenic changes {[}1{]}
\end{quote}

\hypertarget{polar-ocean-observatories-and-marine-monitoring-programs}{%
\subsubsection{Polar ocean observatories and marine monitoring
programs}\label{polar-ocean-observatories-and-marine-monitoring-programs}}

Polar marine monitoring initiatives such as FRAM \ldots{} are working to
gauge the effects of climate change on such rapidly changing
environments.

purl

label

http://purl.obolibrary.org/obo/CHEBI\_17992

Sucrose

http://purl.obolibrary.org/obo/CHEBI\_80716

aplysiatoxin

http://purl.obolibrary.org/obo/CHEBI\_90820

11(R)-HEPE(1-)

http://purl.obolibrary.org/obo/CHEBI\_86386

3-mercaptopropionate

http://purl.obolibrary.org/obo/CHEBI\_17754

Glycerin

http://purl.obolibrary.org/obo/CHEBI\_17754

glycerol

http://purl.obolibrary.org/obo/CHEBI\_16810

2-oxoglutarate(2-)

http://purl.obolibrary.org/obo/CHEBI\_16914

salicylic acid

http://purl.obolibrary.org/obo/CHEBI\_16914

Salicylic Acid

http://purl.obolibrary.org/obo/CHEBI\_16811

Methionine

\hypertarget{fram-hausgarten}{%
\paragraph{FRAM \& HAUSGARTEN}\label{fram-hausgarten}}

At the forefront of climate change affected environments are polar
habitats.

HAUSGARTEN intro: {[}2{]}

FRAM intro: {[}3{]}

\hypertarget{atlantos}{%
\paragraph{AtlantOS}\label{atlantos}}

//maybe mention this?

the Atlantic Ocean Observation Systems (AtlantOS)
\href{https://www.atlantos-h2020.eu/2017/02/10/1st-atlantos-briefing-paper/}{1st
AtlantOS Briefing Paper}

\hypertarget{policy-and-sdgios}{%
\subsection{Policy and SDGIOs}\label{policy-and-sdgios}}

\href{http://journals.plos.org/plosbiology/article?id=10.1371/journal.pbio.1000531}{Making
Marine Life Count: A New Baseline for Policy} {[}4{]} Just use a little
bit from this as policy intro.

\href{http://deepoceanobserving.org/wp-content/uploads/2017/07/DOOS-Consultative-Draft-V5-1-2017-06-19.pdf}{DOOS
Consultative Draft} (no DOI) for insight into functions that can be
understood as ecosystem services of the deep, and thus linked to natural
capital.

\hypertarget{un-sustainability-development-goals-in-response-to-climate-change}{%
\subsubsection{UN sustainability development goals in response to
climate
change}\label{un-sustainability-development-goals-in-response-to-climate-change}}

The effects of increased climate change and extreme weather events are
hardest felt by indigenous people and the global precariat subsiding via
land and ocean subsistence farming and fishing.

\href{https://sustainabledevelopment.un.org/content/documents/21252030\%20Agenda\%20for\%20Sustainable\%20Development\%20web.pdf}{UN
publication: TRANSFORMING OUR WORLD: THE 2030 AGENDA FOR SUSTAINABLE
DEVELOPMENT} no DOI reference for the sustainable development goals and
targets.

The UN framework for SDG's have setup targets for improvements to many
global issues such as UN SDG 14 for ocean health.

14.1

\begin{quote}
By 2025, prevent and significantly reduce marine pollution of all kinds,
in particular from land-based activities, including marine debris and
nutrient pollution
\end{quote}

link the nitrogen phosphorus data to the concept of those cycle being
out of balance as doccumented in the Planetary Boundaries: Exploring the
Safe Operating Space for Humanity paper. {[}5{]}

\hypertarget{need-for-semantics-in-environmental-data}{%
\subsection{Need for semantics in Environmental
data}\label{need-for-semantics-in-environmental-data}}

Observatories generate considerable volumes and varieties of data. The
management and integration of such data remains a major obstacle, as the
data are often not semantically interoperable. I.e. the data cannot be
used in combination, because they are not annotated with a controlled
vocabulary of interconnected terms which would allow for a computer to
perform logical reasoning upon them.

\hypertarget{fair}{%
\subsubsection{FAIR}\label{fair}}

the FAIR data guiding principles (machine-focused findability,
accessibility, interoperability reusability) {[}6{]}

\hypertarget{pangaea}{%
\paragraph{PANGAEA}\label{pangaea}}

observational networks often upload their data to open access
repositories such as the \href{https://pangaea.de/}{PANGAEA}

Although vast quantities of environmental data are freely available to
the scientific community, integrated analysis of such data is hindered
by a lack of logical connections between different types of data.

\hypertarget{linking-earth-science-data-initiatives-such-esip-open-knowledge-network-to-the-un-sdgios}{%
\subsubsection{Linking earth science data initiatives such ESIP Open
knowledge network to the UN
SDGIO's}\label{linking-earth-science-data-initiatives-such-esip-open-knowledge-network-to-the-un-sdgios}}

There exist a variety of earth and life science initiatives attempting
to capture and represent the knowledge associated with environmental
data. \ldots{}

The knowledge required to interface the concepts needed for the
Sustainable development goals are represented in a machine operable form
via the SDGIO sustainable development goals interface ontology.

\hypertarget{role-of-data-in-2015---2020-esip-strategic-plan}{%
\paragraph{\texorpdfstring{role of data in
\href{http://wiki.esipfed.org/index.php/2015-2020_Strategic_Plan}{2015 -
2020 ESIP Strategic
Plan}}{role of data in 2015 - 2020 ESIP Strategic Plan}}\label{role-of-data-in-2015---2020-esip-strategic-plan}}

\href{https://github.com/kaiiam/kblumberg_masters_thesis/wiki/log\#080118}{link
to my log}

\hypertarget{opendap}{%
\paragraph{OPeNDAP}\label{opendap}}

\begin{quote}
OPeNDAP will be a fundamental component of systems which provide
machine-to-machine interoperability with semantic meaning in a highly
distributed environment of heterogeneous datasets.
\end{quote}

\href{https://www.opendap.org/about}{Open-source Project for a Network
Data Access Protocol} There is a need for semantic interoperability
\ldots{}

\hypertarget{knowledge-outreach}{%
\paragraph{knowledge outreach}\label{knowledge-outreach}}

Knowledge graphs are becoming more popular and useful, need to bridge
the gap between patchy but growing resources such as Wikipedia, and
expert knowledge (locked away in text books), using an ontology helps to
bridge this, it can be applied to querying Wikipedia data and for
improved semantic representation make data FAIR. Ontology for an agreed
upon term structure

\hypertarget{ontologies-and-the-obo-foundry}{%
\subsubsection{Ontologies and the OBO
Foundry}\label{ontologies-and-the-obo-foundry}}

Ontology, a human and machine readable semantic representation of domain
knowledge \ldots{}

An ontology is a hierarchically structured, machine and human readable
representation of the knowledge used by experts to describe entities,
and capture the relationships between them {[}7{]}. In informatics,
ontologies exist in the form of a knowledge graph, where nodes represent
entities, and edges represent logical relations linking entities
together (i.e.~axioms). Ontologies provide a digital semantic
infrastructure upon which advanced querying, discovery and analysis of
data can occur.

Ontologies are a methodology to systematically structure and connect
data, allowing users to ask more complicated questions involving the
synthesis of disparate data types which currently can not be combined.

//revise a bit from lab rotation: Because, no single knowledge graph can
encompass the needs of interdisciplinary projects, work must be done in
a coordinated fashion with other ontology researchers and developers. In
order to interconnect ontologies representing scientific knowledge from
different domains, the Open Biological and Biomedical Ontology (OBO)
Foundry and Library was created {[}7{]}. The OBO Foundry and Library
established a set of principles by which to develop and coordinate
ontologies such that the scientific knowledge they represent and hence
the data they link can interoperate. These ontologies share a common
upper level in the hierarchy and use of the same types of logical
connective operations to interlink their knowledge. Following these
principles are a family of ontologies representing scientific knowledge
from non-overlapping domains, which can be used in combination to
describe natural phenomena in greater depth. OBO compliant ontologies
make use of the \href{https://github.com/BFO-ontology/BFO}{Basic Formal
Ontology (BFO)}, to ensure they have a compatible hierarchical
structure, and use logical relations from the
\href{https://github.com/oborel/obo-relations}{Relations Ontology (RO)},
to standardize the connections between their knowledge.

OBO compliant ontologies can be benefit observatory networks such as
Hausgarten FRAM, by providing connections between data collected by
researchers of different disciplines studying overlapping entities.

//example from my rotation add something like this. \textgreater{} For
example sea ice physicists studying the reflectivity of various ice mass
features, may have light intensity data that would help microbial
ecologists studying photosynthetic bacteria in brine channels, to
calculate the light dependent growth rates of such bacteria

\hypertarget{sdgio}{%
\paragraph{SDGIO}\label{sdgio}}

United Nations Environment Programme

SDGIO is an OBO compliant ontology

uses the same interoperable semantic standards to ENVO. Although UNEP
PURLS cannot currently be queried.

\hypertarget{envo-for-representing-environmental-semantics.}{%
\paragraph{ENVO for representing environmental
semantics.}\label{envo-for-representing-environmental-semantics.}}

ENVO papers: {[}8{]} {[}9{]}

The Environment Ontology (ENVO) represents expert knowledge about
different types of environments{[}8{]}{[}9{]}. ENVO is an OBO aligned
ontology.

Environmental knowledge represented by ENVO is used to annotate data
from a variety of life science disciplines including oceanography and
polar research. {[}8{]}{[}9{]}

\hypertarget{gene-ontology}{%
\paragraph{Gene Ontology}\label{gene-ontology}}

go paper: {[}10{]}

GO frequently used to interpret omic data {[}10{]}. It has been used to
do genomewide RNA expression profile data to compare samples based on
shared biological pathways. {[}11{]}

The combination of GO and ENVO is less frequently attempted. {[}12{]}

Paring GO with ENVO is a potential avenue for future study allowing
researchers to ask questions such as \textgreater{} ``What is the omic
potential of microbes associated with particular environments?''.

\hypertarget{example-ontology-uses}{%
\paragraph{Example Ontology uses}\label{example-ontology-uses}}

A communal catalogue reveals Earth's multiscale microbial diversity.
//Uses EMPO a light-weight application ontology built on ENVO the Earth
Microbiome Project Ontology {[}13{]} //good to have an example which
demonstrates the utility of ENVO for an application ontology to provide
utility.

//from my rotation rewrite example \textgreater{} Thesen et al.13. show
how such a federated semantic approach can enhance handling of
environmental and phenotype data, in order to ask increasingly complex
questions such as ``Which crop varieties are expected to do well in a
particular location over the next century?''. Thesen et al
\href{https://www.ncbi.nlm.nih.gov/pmc/articles/PMC4690371/}{Emerging
semantics to link phenotype and environment} {[}14{]}

\hypertarget{competency-questions}{%
\subsubsection{competency questions:}\label{competency-questions}}

In order to leverage growing data and knowledge representation semantic
infrastructure we test if a semantic knowledge web represented by an
ontologies can be used in combination with AWI data to address
competency questions such as:

\begin{center}\rule{0.5\linewidth}{\linethickness}\end{center}

\hypertarget{materials-and-methods}{%
\section{Materials and Methods}\label{materials-and-methods}}

\hypertarget{datasets-used-in-datastore}{%
\subsubsection{Datasets used in
Datastore}\label{datasets-used-in-datastore}}

\begin{enumerate}
\def\labelenumi{\arabic{enumi}.}
\item
  Inorganic nutrients measured on water bottle samples at AWI HAUSGARTEN
  during POLARSTERN cruise MSM29. {[}15{]}
\item
  Physical oceanography and current meter data from mooring TD-2014-LT.
  {[}16{]}
\item
  Chlorophyll a measured on water bottle samples during POLARSTERN
  cruise ARK-XXIV/2. {[}17{]}{[}18{]}
\item
  Global chlorophyll ``a'' concentrations for diatoms, haptophytes and
  prokaryotes obtained with the Diagnostic Pigment Analysis of HPLC data
  compiled from several databases and individual cruises.
  {[}19{]}{[}20{]}
\item
  Biogenic particle flux at AWI HAUSGARTEN from mooring FEVI7.
  {[}21{]}{[}22{]}
\item
  Snow height on sea ice and sea ice drift from autonomous measurements
  from buoy 2015S22, deployed during the Norwegian Young sea ICE cruise
  N-ICE 2015. {[}23{]}{[}24{]}
\item
  Sea ice thickness at Ice Camp 1 on 2013-09-01
  (GEM2IceTh\_DiveHole\_IceStation1). {[}25{]}{[}26{]}
\item
  Ice-algal chlorophyll a and physical properties of multi-year and
  first-year sea ice of core CASIMBO-CORE-1\_10. {[}27{]}{[}28{]}
\item
  //TODO add genomic data preferably some FRAM data from eddie like an
  otu table and or functional genomic table.
\end{enumerate}

\hypertarget{programs-used}{%
\subsubsection{programs used:}\label{programs-used}}

sparql, python, N3, turtle, any23, owl,
\href{https://protege.stanford.edu/}{Protégé}

\hypertarget{semantic-data-annotation}{%
\subsubsection{Semantic Data
Annotation}\label{semantic-data-annotation}}

Semantic annotation of example data was conducted in the RDF
serialization turtle, drawing upon its blank node feature to facilitate
scripting owl code in RDF. Annotations make use ontology terms from the
OBO Foundry {[}7{]}. Ontology terms can be search for using
\href{http://www.ontobee.org/}{Ontobee} A linked data server hosting
ontologies and their terms. {[}29{]}

\hypertarget{sparql-query-scripting}{%
\subsubsection{sparql query scripting}\label{sparql-query-scripting}}

scripts to perform queries were written in python verion?

using the rdf-lib module

Queryies preformed against the ontobee endpoint
http://sparql.hegroup.org/sparql/ a serive provied by the He Group
{[}29{]}

The script makes use of a conjunctive graph object from the rdf-lib
module, to emulate an RDF triple store.

\begin{center}\rule{0.5\linewidth}{\linethickness}\end{center}

\hypertarget{results}{%
\section{Results}\label{results}}

In my masters thesis work I have devised a semantic data annotation and
querying schema. It allows for the phenomena inhering in data, to be
represented and searched in the same way as ontology classes. Annotating
data to be semantically inter-operable with existing ontologies, allows
us to ask questions of interdisciplinary data, making use of the
connections between phenomena encoded within ontologies.

In my masters thesis work I have been writing scripts to assemble and
query a demonstration datastore comprised of semantically annotated AWI
data. As a part of my proposed work, I would create a human and
machine-readable web accessible endpoint to host a variety of AWI data,
as well as a the semantic search tools to facilitate querying it.

\hypertarget{competency-questions-1}{%
\subsection{Competency Questions}\label{competency-questions-1}}

experiments to test knowledge model against competency questions.

\hypertarget{lookup-author-of-ontology-term}{%
\subsubsection{Lookup author of ontology
term}\label{lookup-author-of-ontology-term}}

\href{https://github.com/kaiiam/kblumberg_masters_thesis/wiki/thesis-pieces\#lookup-author-of-ontology-term}{see
my thesis here}

\hypertarget{retrieve-any-data-which-is-about-a-subclass-of-sea-ice}{%
\subsubsection{Retrieve any data which is about a subclass of sea
ice}\label{retrieve-any-data-which-is-about-a-subclass-of-sea-ice}}

//easy to bang out
\href{https://github.com/kaiiam/kblumberg_masters_thesis/wiki/thesis-pieces\#retrieve-any-data-which-is-about-a-subclass-of-sea-ice}{see
my thesis here}

\hypertarget{what-compounds-play-a-role-as-algae-metabolites}{%
\subsubsection{What compounds play a role as algae
metabolites?}\label{what-compounds-play-a-role-as-algae-metabolites}}

easy enough to answer Make use of the CHEBI class:
\href{http://purl.obolibrary.org/obo/CHEBI_84735}{algal metabolite}

purl

querying the \href{http://www.ontobee.org/sparql}{ontobee sparql
endpoint}

\begin{lstlisting}
PREFIX obo: <http://purl.obolibrary.org/obo/>
PREFIX owl: <http://www.w3.org/2002/07/owl#>
SELECT DISTINCT ?purl (STR(?label) as ?label)
WHERE
{
  ?purl rdfs:subClassOf/owl:someValuesFrom obo:CHEBI_84735.
  ?purl rdfs:subClassOf/owl:onProperty obo:RO_0000087.
  ?purl rdfs:label ?label.
}
GROUP BY ?purl
LIMIT 10
\end{lstlisting}

This query gives us the purls and the labels of the first 20 classes
which are subclasses of `has role' some algal metabolite

using the restriction has role.

The group by ?purl is to ensure we don't get duplicates of purls which
have duplicated labels such as
http://purl.obolibrary.org/obo/CHEBI\_15756 which has labels:
\passthrough{\lstinline!hexadecanoic acid!} and
\passthrough{\lstinline!Hexadecanoic acid!}

Returning the following results:

\begin{longtable}[]{@{}ll@{}}
\caption{Compounds serving as algal metabolites.}\tabularnewline
\toprule
purl & label\tabularnewline
\midrule
\endfirsthead
\toprule
purl & label\tabularnewline
\midrule
\endhead
http://purl.obolibrary.org/obo/CHEBI\_17992 & Sucrose\tabularnewline
http://purl.obolibrary.org/obo/CHEBI\_80716 &
aplysiatoxin\tabularnewline
http://purl.obolibrary.org/obo/CHEBI\_90820 &
11(R)-HEPE(1-)\tabularnewline
http://purl.obolibrary.org/obo/CHEBI\_86386 &
3-mercaptopropionate\tabularnewline
http://purl.obolibrary.org/obo/CHEBI\_17754 & Glycerin\tabularnewline
http://purl.obolibrary.org/obo/CHEBI\_17754 & glycerol\tabularnewline
http://purl.obolibrary.org/obo/CHEBI\_16810 &
2-oxoglutarate(2-)\tabularnewline
http://purl.obolibrary.org/obo/CHEBI\_16914 & salicylic
acid\tabularnewline
http://purl.obolibrary.org/obo/CHEBI\_16914 & Salicylic
Acid\tabularnewline
http://purl.obolibrary.org/obo/CHEBI\_16811 & Methionine\tabularnewline
\bottomrule
\end{longtable}

\hypertarget{pco-contributions-plankton-ecology}{%
\subsection{PCO contributions \& Plankton
Ecology}\label{pco-contributions-plankton-ecology}}

//Assuming I get any of this stuff pushed to PCO and or ENVO.

I may still be able to write about the proposed design patterns for PCO,
even if I don't get to submit a pull request.

\hypertarget{tilman-satelite-data}{%
\paragraph{Tilman Satelite Data}\label{tilman-satelite-data}}

Paper: Diatom Phenology in the Southern Ocean: Mean Patterns, Trends and
the Role of Climate Oscillations. {[}30{]} //Associated with the
plankton ecology project using Tillman satellite chlorophyll data and
the plankton bloom ontology classes.

\hypertarget{cryomixs}{%
\subsubsection{cryoMIxS}\label{cryomixs}}

original MIxS paper: {[}31{]}

//talk about my contributions to the cryoMIxS project. Including work
from my lab rotation.

\hypertarget{envo-releases-of-interest}{%
\paragraph{ENVO releases of interest:}\label{envo-releases-of-interest}}

\href{https://github.com/EnvironmentOntology/envo/releases/tag/v2017-05-10}{Ecotone},
\href{https://github.com/EnvironmentOntology/envo/releases/tag/v2017-04-15}{Polar
express},
\href{https://github.com/EnvironmentOntology/envo/releases/tag/v2017-03-27}{Hot
tub time machine}.

\begin{longtable}[]{@{}lll@{}}
\caption{Sample grid table.}\tabularnewline
\toprule
\begin{minipage}[b]{0.20\columnwidth}\raggedright
Fruit\strut
\end{minipage} & \begin{minipage}[b]{0.20\columnwidth}\raggedright
Price\strut
\end{minipage} & \begin{minipage}[b]{0.27\columnwidth}\raggedright
Advantages\strut
\end{minipage}\tabularnewline
\midrule
\endfirsthead
\toprule
\begin{minipage}[b]{0.20\columnwidth}\raggedright
Fruit\strut
\end{minipage} & \begin{minipage}[b]{0.20\columnwidth}\raggedright
Price\strut
\end{minipage} & \begin{minipage}[b]{0.27\columnwidth}\raggedright
Advantages\strut
\end{minipage}\tabularnewline
\midrule
\endhead
\begin{minipage}[t]{0.32\columnwidth}\raggedright
Bananas\strut
\end{minipage} & \begin{minipage}[t]{0.32\columnwidth}\raggedright
\$1.34\strut
\end{minipage} & \begin{minipage}[t]{0.32\columnwidth}\raggedright
\begin{itemize}
\tightlist
\item
  built-in wrapper
\item
  bright color
\end{itemize}\strut
\end{minipage}\tabularnewline
\begin{minipage}[t]{0.32\columnwidth}\raggedright
Oranges\strut
\end{minipage} & \begin{minipage}[t]{0.32\columnwidth}\raggedright
\$2.10\strut
\end{minipage} & \begin{minipage}[t]{0.32\columnwidth}\raggedright
\begin{itemize}
\tightlist
\item
  cures scurvy
\item
  tasty
\end{itemize}\strut
\end{minipage}\tabularnewline
\bottomrule
\end{longtable}

\hypertarget{post-compositional-data-annotation-model}{%
\subsection{post compositional data annotation
model}\label{post-compositional-data-annotation-model}}

//Maybe this could go in material and methods but I'll argue this is a
result in the sense of semantic research, a model for data annotation.

In this work we present a novel semantic data annotation model.
Semantics have been used to represent data \ldots{} //TODO FIND REFS. In
this model data annotations are composed of terms from the OBO Foundry.
Data annotations are written in The RDF turtle specification, and
structured as nested owl classes. Annotating the data as owl classes
ensures parity to the OBO ontologies. This enables us to perform sparql
queries on the annotated data in the same manor as would be done to
query OBO Foundry ontologies.

In order to emulated owl code written in RDF, we chose the turtle RDF
format for its ability to nest blank nodes within strings of triples.

//ADD THE is about property in the data model, it could also be cool to
have a vue figure which explains the workflow.

The creation of ontology classes involves the composition of axioms, the
links between classes, which are assembled from other preexisting
ontology classes and relational properties. In ontology development this
is refereed to as precomposition, which has the effect of taking a set
of ontology classes and properties and joining them together in a
specific way and assigning this assemblage to be a novel class.

The proposed semantic data annotation model allows for this process to
be done in reverse. This is not necessary when an appropriate term for
annotation already exists, however, in cases where the appropriate
annotation term is lacking, it can be created from a combination of
other terms. This practice, referred to as ``post composition'', enables
a user to annotate their data with axioms that comprise a non existent
ontology term. By writing the data annotations as owl classes, they are
functionally equivalent to existing ontology classes, in terms of their
ability to be searched for using a sparql query.

This allows for the phenomena inhering in data, to be represented in a
machine readable semantic layer prior to their incorporation as ontology
terms.

The model makes use of owl equivalence classes, to structure the
annotation as the intersection (and) and or union (or) of post
compositionally annotated classes.

Thus the proposed data annotation model will allow for users, who are
not ontologists, to post compositionally annotate their data. //ADD
section about how I'll write a tool to automate this in the outlook.

\hypertarget{example-of-post-compositional-data-annotation-with-ontology-terms}{%
\subsubsection{Example of Post Compositional Data Annotation with
Ontology
Terms}\label{example-of-post-compositional-data-annotation-with-ontology-terms}}

\href{https://github.com/kaiiam/kblumberg_masters_thesis/wiki/thesis-pieces\#pre-and-post-composition-of-complex-classes}{change
this competency question example} to be about how to annotate data which
is about a \textbf{marine environment determined by a diatom community}
or a \textbf{marine environment determined by a diatom community bloom}
instead of being about I intened to create these classes.

\hypertarget{vocamp-virtual-glacial-hackathon}{%
\subsection{Vocamp Virtual Glacial
Hackathon}\label{vocamp-virtual-glacial-hackathon}}

\href{http://vocamp.org/wiki/Main_Page}{vocamp}:

\begin{quote}
VoCamp is a series of informal events where people can spend some
dedicated time creating lightweight vocabularies/ontologies for the
Semantic Web/Web of Data.
\end{quote}

Virtual-Hackahon-on-Glacier-topic

//to be held on Feb.~2nd. I should have an example of moving snow and
ice related data ready to demonstrate by then.

\hypertarget{awi-dbpedia-contributions}{%
\subsubsection{AWI DBPEDIA
contributions}\label{awi-dbpedia-contributions}}

Contributing semantic knowledge to the website Wikipedia in the form of
an improved heirarchaly structure but aligning with ENVO.

Implement and talk about dpbedia contributions, hopefully they'll let me
edit. My intention is to align dbpedia glacial semantics to those in
ENVO, should be relatively quick and easy once I can edit.

\hypertarget{unep-sdgio}{%
\subsubsection{UNEP SDGIO}\label{unep-sdgio}}

Despite operating within a semantically which is interoperable with the
OBO Foundry the UNEP ontology is currently non queryable. Future work
needs to be done to improve the way SDGIO purls are hosted via UNEP so
that they can be querable. This would allow for the the incorporation of
data mobililzed via semantics to the UN SDGs to help achieve their
objectives.

\begin{center}\rule{0.5\linewidth}{\linethickness}\end{center}

\hypertarget{discussion}{%
\section{Discussion}\label{discussion}}

\hypertarget{querying-semantically-annotated-data}{%
\subsection{Querying Semantically Annotated
Data}\label{querying-semantically-annotated-data}}

using polar semantics to annotate AWI Polar data in a machine-readable
way. This allows for knowledge to be captured in a data querying

\hypertarget{creating-classes-vs-post-compositional-annotation-for-data-annotation}{%
\subsubsection{Creating Classes vs post compositional annotation for
data
annotation}\label{creating-classes-vs-post-compositional-annotation-for-data-annotation}}

\hypertarget{semantics-as-awi-public-outreach}{%
\subsection{Semantics as AWI Public
Outreach}\label{semantics-as-awi-public-outreach}}

AWI
\href{https://www.awi.de/en/science/special-groups/bionics/education-communication.html}{Education
\& Communication}

Contributions to semantic models such as those discussed in this work
serve to improve AWI public outreach efforts to educate and communicate
polar research outputs to the public. Dissemination of AWI knowledge has
been demonstrated in this work via the contributions made to the open
source encyclopedia Wikipedia. This was achieved by aligning the dpbedia
ontology glacial semantics to those of ENVO, which were contributed
during this work.

\begin{center}\rule{0.5\linewidth}{\linethickness}\end{center}

\hypertarget{conclusion}{%
\section{Conclusion}\label{conclusion}}

This work has demonstrated that semantics can be used to mobilize polar
data.

\begin{center}\rule{0.5\linewidth}{\linethickness}\end{center}

\hypertarget{outlook}{%
\section{Outlook}\label{outlook}}

I believe the use of ontologies and semantics data annotation could
serve as a valuable tool to address broad biological questions, such as
those in the Raes et al.~2017 paper, about which mechanism, temperature
or productivity is responsible for marine microbial diversity.

An outlook for the goals presented in this work would be to semantically
annotate a wide variety of interdisciplinary AWI datasets in order
render such data machine-readable and query-able. This creates the
possibility to ask deeper questions of large data sets to address
fundamental biological questions such as: ``Does microbial diversity
coincide with temperature or with primary productivity sourced from
nitrogen fixation?''

Such questions could be asked of semantically annotated and
machine-readable genomic datasets, which contain basic metadata. Such
data could be sourced from anywhere, in house AWI data or already
published data, from a variety environmental locations. Working with a
data publication service such as PANGAEA to host such data in an open
machine-readable web accessible format would allow for complex queries
and questions to be asked.

For example to address the aforementioned question, we would perform a
query to gather all datasets which include temperature, functional
genomic and taxonomic information. From this ecological analysis could
be conducted such as testing if temperature tends to correlate with
microbial diversity, or with samples enriched in nitrogen fixation
genes. The intentional interoperability between the Environment Ontology
and the Gene Ontology would facilitate a query for the latter.

\begin{center}\rule{0.5\linewidth}{\linethickness}\end{center}

\hypertarget{appendices}{%
\section{Appendices}\label{appendices}}

\hypertarget{python-scripts-and-documentation}{%
\subsection{Python Scripts and
Documentation}\label{python-scripts-and-documentation}}

\hypertarget{script-1}{%
\subsubsection{script 1 \ldots{}}\label{script-1}}

\hypertarget{script-2}{%
\subsubsection{script 2 \ldots{}}\label{script-2}}

\begin{center}\rule{0.5\linewidth}{\linethickness}\end{center}

\hypertarget{references}{%
\section*{References}\label{references}}
\addcontentsline{toc}{section}{References}

\hypertarget{refs}{}
\leavevmode\hypertarget{ref-Hutchins_2017}{}%
1. \textbf{Hutchins DA, Fu F}. Microorganisms and ocean global change.
\emph{Nature Microbiology} 2017;2:17058.

\leavevmode\hypertarget{ref-Soltwedel_2005}{}%
2. \textbf{Soltwedel T, Bauerfeind E, Bergmann M, Budaeva N, Hoste E
\emph{et al.}} HAUSGARTEN: Multidisciplinary investigations at a
deep-sea, long-term observatory in the arctic ocean. \emph{Oceanography}
2005;18:46--61.

\leavevmode\hypertarget{ref-Soltwedel_2013}{}%
3. \textbf{Soltwedel T, Schauer U, Boebel O, Nothig E-M, Bracher A
\emph{et al.}} FRAM - FRontiers in arctic marine monitoring visions for
permanent observations in a gateway to the arctic ocean. In: \emph{2013
MTS/IEEE OCEANS - bergen}. IEEE. Epub ahead of print June 2013. DOI:
\href{https://doi.org/10.1109/oceans-bergen.2013.6608008}{10.1109/oceans-bergen.2013.6608008}.

\leavevmode\hypertarget{ref-Williams_2010}{}%
4. \textbf{Williams MJ, Ausubel J, Poiner I, Garcia SM, Baker DJ
\emph{et al.}} Making marine life count: A new baseline for policy.
\emph{PLoS Biology} 2010;8:e1000531.

\leavevmode\hypertarget{ref-Rockstr_m_2009}{}%
5. \textbf{Rockström J, Steffen W, Noone K, Persson, Chapin FSI \emph{et
al.}} Planetary boundaries: Exploring the safe operating space for
humanity. \emph{Ecology and Society};14. Epub ahead of print 2009. DOI:
\href{https://doi.org/10.5751/es-03180-140232}{10.5751/es-03180-140232}.

\leavevmode\hypertarget{ref-Wilkinson_2016}{}%
6. \textbf{Wilkinson MD, Dumontier M, Aalbersberg IJ, Appleton G, Axton
M \emph{et al.}} The FAIR guiding principles for scientific data
management and stewardship. \emph{Scientific Data} 2016;3:160018.

\leavevmode\hypertarget{ref-Smith_2007}{}%
7. \textbf{Smith B, Michael Ashburner, Rosse C, Bard J, Bug W \emph{et
al.}} The OBO foundry: Coordinated evolution of ontologies to support
biomedical data integration. \emph{Nature Biotechnology}
2007;25:1251--1255.

\leavevmode\hypertarget{ref-Buttigieg_2013}{}%
8. \textbf{Buttigieg P, Morrison N, Smith B, Mungall CJ, and SEL}. The
environment ontology: Contextualising biological and biomedical
entities. \emph{Journal of Biomedical Semantics} 2013;4:43.

\leavevmode\hypertarget{ref-Buttigieg_2016}{}%
9. \textbf{Buttigieg PL, Pafilis E, Lewis SE, Schildhauer MP, Walls RL
\emph{et al.}} The environment ontology in 2016: Bridging domains with
increased scope, semantic density, and interoperation. \emph{Journal of
Biomedical Semantics};7. Epub ahead of print September 2016. DOI:
\href{https://doi.org/10.1186/s13326-016-0097-6}{10.1186/s13326-016-0097-6}.

\leavevmode\hypertarget{ref-Ashburner_2000}{}%
10. \textbf{Ashburner M, Ball CA, Blake JA, Botstein D, Butler H
\emph{et al.}} Gene ontology: Tool for the unification of biology.
\emph{Nature Genetics} 2000;25:25--29.

\leavevmode\hypertarget{ref-Subramanian_2005}{}%
11. \textbf{Subramanian A, Tamayo P, Mootha VK, Mukherjee S, Ebert BL
\emph{et al.}} Gene set enrichment analysis: A knowledge-based approach
for interpreting genome-wide expression profiles. \emph{Proceedings of
the National Academy of Sciences} 2005;102:15545--15550.

\leavevmode\hypertarget{ref-Henschel_2015}{}%
12. \textbf{Henschel A, Anwar MZ, Manohar V}. Comprehensive
meta-analysis of ontology annotated 16S rRNA profiles identifies beta
diversity clusters of environmental bacterial communities. \emph{PLOS
Computational Biology} 2015;11:e1004468.

\leavevmode\hypertarget{ref-Thompson_2017}{}%
13. A communal catalogue reveals earth's multiscale microbial diversity.
\emph{Nature}. Epub ahead of print November 2017. DOI:
\href{https://doi.org/10.1038/nature24621}{10.1038/nature24621}.

\leavevmode\hypertarget{ref-Thessen_2015}{}%
14. \textbf{Thessen AE, Bunker DE, Buttigieg PL, Cooper LD, Dahdul WM
\emph{et al.}} Emerging semantics to link phenotype and environment.
\emph{PeerJ} 2015;3:e1470.

\leavevmode\hypertarget{ref-bauerfeind2014inmo}{}%
15. \textbf{Bauerfeind E, Kattner G, Ludwichowski K-U, Nöthig E-M,
Sandhop N}. Inorganic nutrients measured on water bottle samples at AWI
HAUSGARTEN during POLARSTERN cruise MSM29. Epub ahead of print 2014.
DOI:
\href{https://doi.org/10.1594/PANGAEA.834685}{10.1594/PANGAEA.834685}.

\leavevmode\hypertarget{ref-bauerfeind2016poac}{}%
16. \textbf{Bauerfeind E, von Appen W-J, Soltwedel T, Lochthofen N}.
Physical oceanography and current meter data from mooring TD-2014-LT.
Epub ahead of print 2016. DOI:
\href{https://doi.org/10.1594/PANGAEA.861860}{10.1594/PANGAEA.861860}.

\leavevmode\hypertarget{ref-nthig2015camo}{}%
17. \textbf{Nöthig E-M, Bauerfeind E, Metfies K, Simon S, Lorenzen C}.
Chlorophyll a measured on water bottle samples during POLARSTERN cruise
ARK-XXIV/2. Data Set; PANGAEA. Epub ahead of print 2015. DOI:
\href{https://doi.org/10.1594/PANGAEA.855799}{10.1594/PANGAEA.855799}.

\leavevmode\hypertarget{ref-N_thig_2015}{}%
18. \textbf{Nöthig E-M, Bracher A, Engel A, Metfies K, Niehoff B
\emph{et al.}} Summertime plankton ecology in fram straita compilation
of long- and short-term observations. \emph{Polar Research}
2015;34:23349.

\leavevmode\hypertarget{ref-soppa2017gcac}{}%
19. \textbf{Soppa MA, Peeken I, Bracher A}. Global chlorophyll "a"
concentrations for diatoms, haptophytes and prokaryotes obtained with
the Diagnostic Pigment Analysis of HPLC data compiled from several
databases and individual cruises. Data Set; PANGAEA. Epub ahead of print
2017. DOI:
\href{https://doi.org/10.1594/PANGAEA.875879}{10.1594/PANGAEA.875879}.

\leavevmode\hypertarget{ref-Losa_2017}{}%
20. \textbf{Losa SN, Soppa MA, Dinter T, Wolanin A, Brewin RJW \emph{et
al.}} Synergistic exploitation of hyper- and multi-spectral precursor
sentinel measurements to determine phytoplankton functional types
(SynSenPFT). \emph{Frontiers in Marine Science};4. Epub ahead of print
July 2017. DOI:
\href{https://doi.org/10.3389/fmars.2017.00203}{10.3389/fmars.2017.00203}.

\leavevmode\hypertarget{ref-bauerfeind2009bpfa}{}%
21. \textbf{Bauerfeind E, Nöthig E-M, Beszczynska A, Fahl K, Kaleschke L
\emph{et al.}} Biogenic particle flux at AWI HAUSGARTEN from mooring
FEVI7. Data Set; PANGAEA. Epub ahead of print 2009. DOI:
\href{https://doi.org/10.1594/PANGAEA.714844}{10.1594/PANGAEA.714844}.

\leavevmode\hypertarget{ref-Bauerfeind_2009}{}%
22. \textbf{Bauerfeind E, Nöthig E-M, Beszczynska A, Fahl K, Kaleschke L
\emph{et al.}} Particle sedimentation patterns in the eastern fram
strait during 20002005: Results from the arctic long-term observatory
HAUSGARTEN. \emph{Deep Sea Research Part I: Oceanographic Research
Papers} 2009;56:1471--1487.

\leavevmode\hypertarget{ref-nicolaus2015shos}{}%
23. \textbf{Nicolaus M, Itkin P, Spreen G}. Snow height on sea ice and
sea ice drift from autonomous measurements from buoy 2015S22, deployed
during the Norwegian Young sea ICE cruise N-ICE 2015. Data Set; Alfred
Wegener Institute, Helmholtz Center for Polar; Marine Research,
Bremerhaven; PANGAEA. Epub ahead of print 2015. DOI:
\href{https://doi.org/10.1594/PANGAEA.846861}{10.1594/PANGAEA.846861}.

\leavevmode\hypertarget{ref-nicolaus2017shaa}{}%
24. \textbf{Nicolaus M, Hoppmann M, Arndt S, Hendricks S, Katlein C
\emph{et al.}} Snow height and air temperature on sea ice from Snow Buoy
measurements. Epub ahead of print 2017. DOI:
\href{https://doi.org/10.1594/PANGAEA.875638}{10.1594/PANGAEA.875638}.

\leavevmode\hypertarget{ref-ricker2017sita}{}%
25. \textbf{Ricker R, Krumpen T, Schiller M}. Sea ice thickness at Ice
Camp 1 on 2013-09-01 (GEM2IceTh\(\_\)DiveHole\(\_\)IceStation1). Data
Set; PANGAEA. Epub ahead of print 2017. DOI:
\href{https://doi.org/10.1594/PANGAEA.870689}{10.1594/PANGAEA.870689}.

\leavevmode\hypertarget{ref-Arndt_2017}{}%
26. \textbf{Arndt S, Meiners KM, Ricker R, Krumpen T, Katlein C \emph{et
al.}} Influence of snow depth and surface flooding on light transmission
through antarctic pack ice. \emph{Journal of Geophysical Research:
Oceans} 2017;122:2108--2119.

\leavevmode\hypertarget{ref-lange2015icaa}{}%
27. \textbf{Lange BA, Michel C, Beckers J, Casey JA, Flores H \emph{et
al.}} Ice-algal chlorophyll a and physical properties of multi-year and
first-year sea ice of core CASIMBO-CORE-1\(\_\)10. Data Set; PANGAEA.
Epub ahead of print 2015. DOI:
\href{https://doi.org/10.1594/PANGAEA.842359}{10.1594/PANGAEA.842359}.

\leavevmode\hypertarget{ref-Lange_2015}{}%
28. \textbf{Lange BA, Michel C, Beckers JF, Casey JA, Flores H \emph{et
al.}} Comparing springtime ice-algal chlorophyll a and physical
properties of multi-year and first-year sea ice from the lincoln sea.
\emph{PLOS ONE} 2015;10:e0122418.

\leavevmode\hypertarget{ref-Ong2017}{}%
29. \textbf{Ong E, Xiang Z, Zhao B, Liu Y, Lin Y \emph{et al.}} Ontobee:
A linked ontology data server to support ontology term dereferencing,
linkage, query and integration. \emph{Nucleic Acids Res}
2017;45:D347--D352.

\leavevmode\hypertarget{ref-Soppa_2016}{}%
30. \textbf{Soppa M, Völker C, Bracher A}. Diatom phenology in the
southern ocean: Mean patterns, trends and the role of climate
oscillations. \emph{Remote Sensing} 2016;8:420.

\leavevmode\hypertarget{ref-Yilmaz_2011}{}%
31. \textbf{Yilmaz P, Kottmann R, Field D, Knight R, Cole JR \emph{et
al.}} Minimum information about a marker gene sequence (MIMARKS) and
minimum information about any (x) sequence (MIxS) specifications.
\emph{Nature Biotechnology} 2011;29:415--420.

\end{document}
